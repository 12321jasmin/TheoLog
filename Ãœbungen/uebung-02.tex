\documentclass[german]{latteachCD}[2017/03/28]

\usepackage[sheetnumber=2]{theolog}

% LOOP- und WHILE-Berechenbarkeit

\begin{document}

\maketitle

\begin{mdframed}
  Die folgenden Aufgaben werden nicht in den Übungen besprochen und dienen der
  Selbstkontrolle.

  \renewcommand{\theexercise}{\Alph{exercise}}
  \setcounter{exercise}{2}

  \begin{exercise}
    Zeigen Sie, dass $\{1\}^{*}$ unentscheidbare Teilmengen besitzt.
  \end{exercise}

  \begin{exercise}
    Welche der folgenden Aussagen sind wahr? Begründen Sie Ihre Antwort.
    \begin{enumerate}
    \item Jedes LOOP-Programm terminiert.
    \item Zu jedem WHILE-Programm gibt es ein äquivalentes LOOP-Programm.
    \item Die Anzahl der Ausführungen von $P$ in der LOOP-Schleife
      \begin{center}
        \ttfamily LOOP $x_{i}$ DO $P$ END
      \end{center}
      kann beeinflusst werden, indem $x_{i}$ in $P$ entsprechend modifiziert wird.
    \item Die Ackermannfunktion ist total und damit LOOP-berechenbar.
    \end{enumerate}
  \end{exercise}
\end{mdframed}

\vspace*{\baselineskip}

\setcounter{exercise}{0}

\begin{exercise}
  Zeigen Sie, dass folgende Funktionen $f\colon \mathbb N^{2} \to \mathbb
  N$ LOOP-berechenbar sind:
  \begin{enumerate}
  \item $f(x,y) \coloneqq \max(x - y, 0)$
  \item $f(x,y) \coloneqq x \cdot y$
  \item $f(x,y) \coloneqq \max(x,y)$
  \item $f(x,y) \coloneqq \operatorname{ggT}(x,y)$, wobei
    $\operatorname{ggT}(x,y)$ den größten gemeinsamen Teiler von $x$ und $y$ bezeichnet.
  \end{enumerate}
\end{exercise}

\begin{exercise}
  Geben Sie ein WHILE-Programm an, das die Funktion $f \colon \mathbb N^{2} \to
  \mathbb N, (x_{1}, x_{2}) \mapsto \operatorname{kgV}(x_{1}, x_{2})$
  berechnet und erklären Sie seine Arbeitsweise.
\end{exercise}

\begin{exercise}
  % Siehe https://jeremykun.com/2012/04/21/kolmogorov-complexity-a-primer/
  Es sei $\Sigma$ ein fest gewähltes Alphabet mit mindestens zwei Elementen.
  Wir betrachten eine Programmiersprache $L$ über $\Sigma$, die in der Lage ist,
  Turing-Maschinen zu simulieren.  Für ein Wort $w \in \Sigma^{*}$ ist die
  \emph{Kolmogorov-Komplexität} $K_{L}(w)$ die Länge des kürzesten Programms in
  $L$, welches bei leerer Eingabe das Wort $w$ als Ausgabe produziert.

  Zeigen Sie folgende Aussagen:
  \begin{enumerate}
  \item Es gibt für jedes $n \in \mathbb N$ ein Wort $w \in
    \Sigma^{*}$ mit $\lvert w\rvert = n$ und $K_{L}(w) \geq n$.
  \item Es gibt eine Konstante $c \in \mathbb N$, so dass gilt: ist $w$ das
    Ergebnis der Berechnung einer Turing-Maschine $M$ mit Eingabe $x$, dann
    \begin{equation*}
      K_{L}(w) \leq \lvert\langle M,x\rangle\rvert + c.
    \end{equation*}
  \item Die Abbildung $w \mapsto K_{L}(w)$ ist unberechenbar.
  \end{enumerate}

\end{exercise}

\end{document}
