\documentclass[german]{latteachCD}[2017/03/28]
\usepackage[sheetnumber=1]{theolog}

\begin{document}

\maketitle

\begin{mdframed}
  Die folgenden Aufgaben werden nicht in den Übungen besprochen und dienen der
  Selbstkontrolle.

  \renewcommand{\theexercise}{\Alph{exercise}}

  \begin{exercise}
    Wiederholen Sie die Begriffe \emph{Einband Turing-Maschine}, \emph{Mehrband
      Turing-Maschine}, \emph{Entscheidungsproblem}, \emph{Unentscheidbarkeit},
    \emph{Aufzählbarkeit}, \emph{Abzählbarkeit} und \emph{Halteproblem}.%
  \end{exercise}

  \begin{exercise}
    Zeigen Sie: Wenn es möglich ist, für zwei beliebige Turing-Maschinen zu
    entscheiden, ob sie dieselbe Sprache akzeptieren, so ist es auch möglich,
    für beliebige Turing-Maschinen zu entscheiden, ob sie die leere Sprache
    akzeptieren.
  \end{exercise}

\end{mdframed}

\vspace*{\baselineskip}

\setcounter{exercise}{0}

\begin{exercise}
  Zeigen Sie folgende Aussagen:
  \begin{enumerate}
  \item $\abs{\mathbb N} = \abs{\mathbb N \times \mathbb N}$;
  \item $\abs{\mathbb N} = \abs{\mathbb Q}$;
  \item $\abs{\mathbb N} \neq \abs{\mathbb R}$;
  \item für jede nicht-leere endliche Menge $\Sigma$ ist $\Sigma^{*}$ abzählbar
    unendlich.
  \end{enumerate}
\end{exercise}

\begin{exercise}
  Sei $M$ eine Menge.  Zeigen Sie, dass es keine surjektive Funktion $f \colon M
  \to \subsets{M}$ gibt.  Folgern Sie daraus, dass stets $\abs{M} <
  \abs{\subsets{M}}$ gilt.
\end{exercise}

\begin{exercise}
  Konstruieren Sie eine Turing-Maschine $\mathcal{A}_{\text{mul}}$, welche die
  Multiplikation zweier natürlicher Zahlen implementiert.  Dabei sollen sowohl
  die Eingaben als auch die Ausgabe unär kodiert sein.
\end{exercise}

\begin{exercise}
  Zeigen Sie: Wenn es möglich ist, für zwei beliebige Turing-Maschinen zu
  entscheiden, ob sie dieselbe Sprache akzeptieren, so ist es auch möglich, für
  beliebige Turing-Maschinen zu entscheiden, ob sie auf der leeren Eingabe
  halten.
\end{exercise}

\end{document}
